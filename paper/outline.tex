1. Introduction   

% Waking up the sleeping dragon. The little Hobbit Describing the dragon.
% Drache, der alle dreissig Jahre auftaucht, seinen heissen Atem verbreitet und wieder einschläft.   

Dealing with a high risk technology such as a nuclear power plant is like dealing with a sleeping dragon in an golden cave. \footnote{Reading the little Hobbit from Tolkin, describing Smaug the dragon sleeping in the mountain covered with gold.}  Don't wake him up, do not disturb his sleep - keep him calm, cool him down, covered with gold. Once they are awake you can't control him so easy. They start to spread their poisoned air.  

 
The use of modern technologies shows that modern societies are more vulnerable because of growing dependencies and interdependencies (Jaeger et al 2001 p. 9). Technologies like nuclear power are essential to provide the power we need to build further wealth and keep economy growing. The price of technology is a society that has to deal with technologies unintended risk's or as Ulrich Beck calls it a life in the risk society (1992). And in fact, even if life improved in the past in many ways (Freudenburg 2001 \citep{Freudenburg:2001cs}), the consequences of our technological improvements for future generation remain unsolved. 

The accident in Fukushima 2011 made clear that the accident in Chernobyl 1986 was not an isolated event unlikely to recur but showed that the occurrence of an nuclear accident is always possible. Even the Fukushima disaster was a man-made disaster for the authorities failed to prevent safety requirements, because with a good safety culture the earthquake and the tsunami where foreseeable UMSTÄNDE und man hätte Schutzmassnahmen aufnehmen können  (Pidgeon 2012 \citep{Pidgeon:2012ud})  (Poortinga et al, 2013 \citep{Poortinga:2013gt}, Funabashi and Kitazawa, 2012 \citep{Funabashi:2012fs}). 

% Besonderheit nuclear Accidents
Nuclear accidents are perceived as very dangerous because they combine all undesirable characteristics of a hazard. They are uncontrollable, caused by others, can happen anytime with no time to prepare, there is not so much knowledge how to protect the BEVÖLKERUNG after an accident. (Slovic et al (1981) ( \citep{Slovic:1981fe}; Renn and Zwick (1997) (\citep{Renn:1997tx, Slovic:1981fe}).   


% Research Question
Our research question is: What are the factors influencing individuals risk perception. Do we observe a risk-gap in the society. How did the accident change the impact of social factors? Do we observe a common sense of risk among people in different countries? We think that social status is an important factor in shaping individual's risk perception (Pampel 2011 \citep{Pampel:2011cx}). Given the specific characteristic of the risk [BEGRIFF: Hochrisikotechnologie], trust authority plays an important role in forming someones risk perception. 

% In this paper we want to examine if the change in risk perception happened equally in the society. The main question therefore is: How did the accident change individual's risk perception? Did the accident divided societies so they became more extreme or did the incident lead to a more homogenized risk perception within societies?



% Studies on impact of nuclear accidents (Fukushima and Chernobyl) 

2. Previous research on nuclear risk perception and the impact of nuclear accidents.

Risk perception  of nuclear power has been in the focus for social sciences not only since the nuclear accident of Chernobyl 1986 and Fukushima 2011. In social science and psychology risk perception and risk assessment of different technologies has been widely investigates (Slovic 2000, Renn 2008). Research on acceptance and risk perception of nuclear power also addresses problem of nuclear waste management (Svenson and Karlsson, 1989; \citep{Svenson:1989up}; Rosa et al, 1993;  ) and preferences for nuclear power in comparison to other energy sources (Truelove et al, 2013; Pidgeon et al, 2008;). Furthermore studies investigate how sociodemographic factors, values, worldviews and norms influence individual's risk judgement on nuclear power (e.g. Pampel, 2011; Slovic et al 1982; Whitfield et al, 2009; de Groot et al, 2012)    
 
Several studies, mostly national samples, have been analyzing the impact of Fukushima's nuclear accident on risk perception or acceptance of technology. Visschers and Siegrist (2012) and Visschers and Wallquist (2013) \citep{Visschers:2012bf,Visschers:2013ee}  report a negative impact on acceptance of nuclear power after the Fukushima accident for Switzerland. A survey conducted in Britain and Japan (Poortinga et al, 2013) \citep{Poortinga:2013gt} does not show any change in people's acceptance of nuclear power in Britain but lower acceptance in Japan. Prati and Zanti (2013) \citep{Prati:2013jc} report lower trust in technology and claim that an accident has the potential to change value orientation especially towards more altruistic values.
Siegrist et al (2014) \citep{Siegrist:2014ji} show that people with an increased risk perception after the accident have lower acceptance of nuclear power in Switzerland. 
Earlier studies after the accidents in Chernobyl (Ukraine, 1986) or on Three Mile Island (USA, 1979) conclude that after the accident acceptance of nuclear power is decreasing (Eiser et al., 1990, Rosa and Freudenburg, 1993) \citep{RichardEiser:1990iw, Rosa:1993uj}. Studies analyzing the long-term change in individuals attitudes towards nuclear power indicates that after the immediate short-term decrease the acceptance of nuclear power increases again towards the pre-accident level (Renn 1990) \citep{Renn:1990kf}.  


An accident like the Fukkushima disaster is an unexpected event, resulting in an change in individual's risk perception. From a theoretical point of view we want to BEITRAGEN to understand the mechanism and processes that shape risk judgements in case of an accident. 



%%%%%%%%%%%%%%%%%%%%%%%             
                                                       

 
3. Conceptualizing of risk and risk perception. 

Because  humans are embedded in an uncertain environment risk is a term that defines the possibility of  uncertain outcomes.  ``Risk is defined as the possibility that an undesirable state of reality (adverse effects) may occur as a result of natural events or human activities'' (Kates et al, 1985 p.21, cited in Renn, 2008 p.98; Rosa, 1998). Risk form this point of view implies that the outcome has a negative impact for something of human value and is not related to an opportunity of an desired outcome (Jaeger et al 2001).

Risk evaluating process implies an anticipation of negative future outcomes considering that the decision is based on limited information of uncertain outcomes. ``Risk perception, in general, denotes the processing of physical signals and/or information about potentially harmful events or activities, and the formation of a judgement about seriousness, likelihood and acceptability of the respective event or activity.''  (Renn, 2008 p 98) \citep[98]{Renn:2008wq}. Individual's risk-perception, is a subjective evaluation of probability and outcomes since the judgement of the (negative) consequences and the probabilistic judgement depend on various factors based on experience, knowledge, emotions and values (WER SAGT DAS? SLOVIC, JAEGER??).        
 
To model change in risk perception more formally we describe the functional relationship between probability of the event or situation and its individual utility. 
Form an economic perspective a rational actor evaluates alternative options and chooses the alternative that maximize its subjectively expected utility (Jaeger et al 2001 \citep{Jaeger:2001wv}; Renn, 2008 (Chapter 1) \citep{Renn:2008wq}). Because risk in our definition is related as negative outcome, risks are perceived as expected utility losses that result from an event or human activity.
  
Individual's expected utility ($E(U)$) is a functional form of a negative outcome ($X$) expressed as utility ($U(X)$) weighted by an uncertainty component expressed as a probability ($p$) that the outcome will occur.  

\begin{equation}
   E(U) =  p \cdot U(X)  \nonumber
\end{equation}
The probability is in the range between 0 and 1 ($0 \leq p \leq 1$). It also implies that the compliment of event $X$  will occur with probability $1-p$.  

The rational actor paradigm  implies that people maximize their utility ($E(U)$). It does not specify how people evaluate their probabilities or how they measure utilities. We assume that individuals base their probability judgement on available informations. Hence to specify the utility process we include the information in the model. The probability (p) of an uncertain event (X) is conditioned on the information (e): p(X|e) (Morgan and Henrion , 1990 p49) ( Fischoff 1981, \citep{Fischhoff:1981tx} [habe ich von Renn 2008 p19]). The information gathering and the probabilistic generating process differs between individuals. We therefore talk of individual or subjective probabilities, because probabilities are based on the current knowledge of each individual. Hence subjective probabilities in this sense are the degree of belief in the occurrence of an event, that a person has at a certain time with a given set of information  (de Finetti, 1974 \citep{deFinetti:1974ua}).   
Assuming that a nuclear disaster has the same negative outcome for everybody (X(nuclear accident)) risk perception can differ in society, because people may have different knowledge or information about the occurrence of an uncertain event. The same person may also change his' or her's probabilities over time if new informations are available. Even if the model is a simplification of the real risk evaluating processes it is one way to explain the change in risk perception due to new available information. 



4. Change in individual's risk perception

After the Fukushima accident the subjective probability (p(X|e)), that an nuclear accident will happen in the future is based on the past knowledge of two major accidents since 1986 (p(X|Fukushima and Chernobyl). This a-posteriori probability should be higher than the a-priori knowledge of only one accident p(X|Chernobyl). Even only a small change in subjective probability will result of a change in risk perception because nuclear power is perceived as a technology with a high catastrophic potential for losses of life (Slovic, 2000, Chapter 8 p 148   \citep[148]{Slovic:2000tx}). A nuclear accident is a signal for a potential thread in the future and has high social impact, leading to a change individuals perceived probability of an accident in the future. The functional relationship (probability x outcome) can change drastically if the  probability changes and the negative outcome is still perceived as very high.  

People also use heuristics to simplify the evaluation of a negative outcome. Not only that individuals have difficulties to interpret low probabilities to make decisions (Kunreuther et al, 2001 \citep{Kunreuther:2001dj}),   individuals also simplify and base there judgement primarily on easy available information. According to the availability heuristic  (Tversky and Kahnemann, 1974; Kahnemann and Tversky, 1979; \citep{Tversky:1974wi,Tversky:1973ui, kahnemann_prospect_1979}) people overestimate the probability of a similar event if a event just happened and is easy to remember. Furthermore the evaluation process can be influences by emotions. According to the affect heuristic if people have strong negative feeling related to a situation or hazard they overestimate its probability (Slovic et al, 2004; Slovic, 2000 \citep{Slovic:2004hj}\citep{Slovic:2000tx}) (Finucane et al, 2000 \citep{Finucane:2000wu}). Emotions become more important by judging a risk, if people don't have enough opportunities to analyze the situation. Both heuristics bear evidence that after an accident people should have a higher risk perception because through media coverage the information are easy available and people are more aware of the negative consequences. This is also true even if people are not affected directly because they live geographically apart.          



For the functional form of risk perception we furthermore assume that with additional information after an accident there is no linear but a concave change in risk perception (BEGRÜNDUNG XXXX LITERATUR?).  That means that the change in risk perception after the accident depends on individual's level before the accident. If someone was convinced that a nuclear accident will happen anytime soon (high a-priori probability) the actual accident in Japan has not much influence to increase the already strong opinion. The new a-posteriori probability again expresses the opinion that an accident can happen anytime soon. For someone who had a low a-priori probability assumption of an accident and believed, that an accident is unlikely, Fukushima's accident now is the prove that nuclear accidents are very likely to happen. A person with a low risk-perception can therefore adjust his' or her's probability assumption more drastically. Our hypothesis therefore is people with on average a higher risk perception before the disaster should have a lower increase  compared to people that had on average a lower risk perception before the accident. 
We call this non-linear change in risk-perception the adaptation-hypotheses due additional negative informations.

From our point of view all described subjective risk-evaluating processes, like updated subjective probability, easily available information and an emotional shock indicate an increase in people's risk perception after the Fukushima accident. 


% WO DAZU??
% It is also the case that there are benefits from the technology that influences individual's risk-perception, like a `clean' image of nuclear power because of low CO2 emissions or a independent source of energy for the country. Individuals accept risks if the corresponding benefit adds more utility than the risk detracts from the utility (Renn, 2008 p.18f [CHECK REF AGAIN]). (Siegrist 2014 oder ein Paper seiner Kolleginnen)                                      

     
5. Explaining differences in individual's risk perception. 

Research shows that  individuals's risk judgement differs systematically between individuals and lead to a heterogeneous risk-perception within society [within society=INNERHALB DER GESELLSCHAFT??]. Gender is strongly related to risk judgement. There is strong empirical evidence that women perceive risks higher and more problematic than men (SLOVIC 2000, 396 \citep[396]{Slovic:2000tx}; de Groot et al, 2012 \citep{deGroot:2012fg}). Women tend to express greater concern than do men toward technology because they are more concerned about consequences for health and safety (Davidson and Freudenburg, 1996 \citep{Davidson:1996uk}).  Males also associate nuclear power rather with neutral or positive attributes like energy production and necessity for economic growth, whereas women rather associate negative emotions like severe accidents and environmental problems (Keller et al, 2011 \citep{Keller:2011gb}; Siegrist, 2014 \citep{Siegrist:2014ji};).  

There is no clear evidence that age affects risk perception. Pampel (2011) \citep{Pampel:2011cx} report a higher support for nuclear energy for older people across European countries, whereas Siegrist et al (2014) report a negative correlation for acceptance of nuclear energy [GIBT ES NOCH ANDERE AUSSAGEN GERADE ZU RISK PERCEPTION NICHT AKZEPTANZ?]. [BRAUCHEN WIR EINE ERKLÄRUNG?? KOHORTEN EFFEKT ÄLTERE VERTRAUEN TECHNIK MEHR, WEIL IM TECHNOLOGISCHEN FORTSCHRITTSGEDANKEN IN DEN 60ER JAHREN GROSS GEWORDEN.] Siegrist et al (2014) \citep{Siegrist:2014ji} report that after the accident older people had a higher chance to change from promoting nuclear technology to opposing nuclear power than younger people.  [GIBT ES ANDERE QUELLEN?] From a rational actor perspective we would assume that older people should have a lower risk perception because the chance that they will experience an accident in the future declines the longer they live. 
                      
Socio-economic status seems to influence individual's risk perception negatively. The inverse relationship indicates that higher educated people as well as higher income [HABEN WIR INCOME VARIABLE AUFGENOMMEN?] class people tend to have a lower risk perception (Pampel, 2011; Greenberg 2009; Greenberg/Truelove 2011 \citep{Pampel:2011cx,Greenberg:2009fx,Greenberg:2011ja}).  One explanation is that better educated people believe that they have a better knowledge and are more familiar with nuclear technology (Pampel 2011). One explanation by Flynn et al. (1994) \citep{Flynn:1994dn} follows a socio-political argument claiming that better educated people    (especially white man) judge high risk technologies like nuclear power as less dangerous because this group of people helped to develop and establish nuclear power as energy source in the past and their energy intense lifestyle has been benefitting the most from cheap energy supply. Another study states that lack of perceived knowledge also results in less support for nuclear power  (Costa-Font et al, 2008 \citep{CostaFont:2008hf}).  The authors conclude that, given the complexity of the technological system, individuals' risk evaluating process is mostly driven rather by intuition, based on trust, than of a risk assessment based only on real facts.  

% adaptation-hypotheses 
According to our theoretical argument that a change in risk-perception follows a non-linear trend after additional negative information of an accident, the change in risk-perception should be higher for individuals with on average lower risk perception.  That means, male and people with higher social economic status, as well as older people should change their risk perception more drastically. 
 
  
Given that nuclear risks are perceived as uncontrollable, catastrophic, involuntary and threatening to future generations (Slovic 2000, p 141)  social trust is an important factor that influences perceived risk. (Keller et al, 2011; Whitfield et al, 2009; Poortinga and Pidgeon, 2003; Greenberg 2009; \citep{Whitfield:2009ku,Poortinga:2003cb,Keller:2011gb, Greenberg:2009fx}). Trust in political institutions and authorities is crucial to believe that governmental institution are able to manage and control present technological risks and to provide protection for those at risk in the future (Kasperson, 2008 p345 \citep{Kasperson:2008tw}).  
Nuclear accidents are destroying trust. Trust is created rather slowly and once lost extremely hard to rebuild. Negative events weight much more than positive events in building or destroying trust in technology. This effect is called the the ``asymmetry principle''. (Slovic, 1993 \citep{Slovic:1993gm}). [SAME article as book chapter Slovic 2000, p 320)].  
After the Fukushima accident people who still report to trust the government, should express a lower risk perception compared to people that did not trust the government. After the accident the reporting in the media about the negative consequences are very obvious, also the lack of scientific knowledge and technical details of how to solve the problem. If after the Fukushima event someone still trust the government, that also means, that they belief the government proved their ability to manage the situation well and will protect them in the future. So the effect should get stronger after the disaster.

    
% if we want to add political orientation: People who express less risk perception also are more conservative.[LITERATURANGABEN DAZU im Sammelbandbeitrag von mir]

6. Data and Methods 

% ISSP Dataset
% Operationalisierung der Variablen 
% Daten in Stichprobe vor und nach dem Unfall. Tabelle mit Fallzahl spezifizieren. 

% Fixed Effects Model
%   Kontrollvariablen: Zeit und Land

% Für die Prüfung der Hypothesen verwenden wir Daten des International Social Survey Programmes (ISSP) 2010 Environmental Module (QUELLE ISSP EINFÜGEN). Die Daten wurden in XX Ländern im Zeitraum von XX bis XX erhoben (Für eine genaue Darstellung der Erhabungszeiträume siehe Abbildung XX). In allen Befragungsländern wird ein einheitlicher Fragebogen mit 60 Umweltspezifischen Fragen gestellt. Der Fragebogen wird meist im face-to-face CAPI interviews geführt, aber auch teilweise mit CATI-Telefoninterviews  per Fragebogen erhoben [GUT WÄRE HIER EINE TABELLE MIT GENAUEM BEFRAGUNGSZEITRAUM; INTERVIEWMETHODE; BEFRAGTE VOR UND NACH FUKUSHIMA]. In der Stichprobe vor dem 11.3. 20111 befinden sich XX Befragte aus XX Ländern. In der Stichprobe nach dem 11.3.2011 befinden sich XX Befragte aus XX Ländern. In XX Länder fällt der Unfall innerhalb des Befragungszeitaum, so dass hier Personen vor und nach dem Unfall befragt wurden [BRACHEN WIR DIE INFO?]. In die Analysen fliessen nur diejenigen Befragten ein, die sicher entweder vor oder nach dem Unfall eingeordent werden können. Befragte aus Canada, die im MÄrz befragt wurden, werden aus den Analysen ausgeschlossen, da nur Monatsgenaue angaben über den Befragungszeitraum vorliegen [WERDEN NOCH WEITERE AUSGESCHLOSSEN].

% Das Risikobewusstsein wird durch die Frage: XX operationalisiert. Befragte können auf einer fünfstufigen Likertskala angeben ob sie XX .... XX gefährlich finden. Der Wertebereich der Variable beträgt daher 1 bis 5. 
% [Tabelle XX mit Variablen-Kennzahlen einfügen]
% Die Soziodemografischen Variablen sind folgendermassen operationalisiert. Geschlecht ist eine Dummy-Variable mit Männlich als Referenzkategorie. Das Alter ist eine metrische Variable im Altersbereich von XX bis XX. 
% Die Bildung einer Person ist in 6 Kategorien erhoben mit ``keine Bildung'' als Referenzkategorie. 
% Vertrauen ist eine Variable die auf  einer [??] Likertskala erhoben wurde. Personen können zum einen angeben, ob sie den PErsonen allgmeien Vertrauen (Allgemeins Vertrauen) und ob sie Menschen Fair finden. [GENAUE OPERATIONALISIERUNG NOCH NACHSCHAUEN] 
% [VERTRAUEN IN REGIERUNG AUCH NOCHMAL GENAU ANSEHEN].

% ANALYEMEHTEODE  


7. Results 
FIGURE AIRPLANEPLOT ABOUT HERE

To test the hypotheses we compare two regression models. The first model only includes respondents [BEFRAGTE?] that were conducted before March 11th, 2011. The second model includes respondents that were asked after the accident.   
Fixed effects models of both models, as displayed in Figure XX (see also for exact figures Table XX in the appendix) confirm the general hypothesis that social factors influence risk perception systematically. Furthermore, we can prove the hypothesis, that after the accident people's risk perception is more homogenized .   


Figure XX shows the relationship between risk perception and sociodemographic factors as explained in Chapter XX [Data and Methods] for respondents before and after the Fukushima accident. The regression coefficient, depicted as a point, describes the change in risk perception if the independent variable changes one unite. The horizontal line describes the 95 Percent confidence interval. If the confidence interval includes the horizontal zero line, there is no significantly different relationship. 
Positive effects indicate a higher risk perception, negative effects a lower risk perception. After the accident, if the positive or negative effect is closer to zero than before the accident, that indicates that risk perception in society got more homogenized after the accident.   


In both sample, as assumed, women have a higher risk perception. The gender effect for women is not as high anymore after the accident. The significant decline indicates that after the accident men's risk perception increased more drastically than women. [T-TEST PROOF of gender level before and after 3/11] Women still have a higher risk perception but the effect is not so strong anymore - evidence that the accident led to a more unified [VEREINHEITLICHTE] risk perception.  

Before 3/11 risk perception decreased with older age [HÖHEREM ALTER] (per 10-years XX index-points). There is no significant age effect for the post accident sample. We therefore find evidence that beside men also older people adapt their risk perception stronger [?? etwas stärker anpassen: to adapt stronger??]   


Education, an indicator for social status, has the expected negative effect on risk perception for better educated people. Before the accident, compared with the reference category ``no formal education'', people with an at least ``higher secondary''  education have a lower nuclear risk perception. E.g. individuals with an ``university degree'' have a XX points lower risk perception. The effect decreases after the accident and is not different from the reference category anymore. This is also evidence in favor of the [adaptaion-after-accident-] hypothesis.


% VERTRAUEN _ GENAUE OPERATIONALSIERUNG NOCHMALS CHECKEN _ OB METRISCH ODER DUMMY: 
Both effects of general trust and trust in governance are negative indicating that trust decreases risk perception towards nuclear power. Interestingly enough the effect of trust changes differently after the accident. On the one hand people who tend to trust people more still have a lower risk perception but this effect is not as clear anymore. On the other hand, people who trust government more after the accident now have an even lower risk perception than the individuals trusting the government before the accident. The increased negative relationship indicated that people who trust the government got more radical and compared to other people perceive nuclear power as less dangerous. [ES IST EINE RELATIVE VERÄNDERUNG ODER IM MITTEL KÖNNTE AUCH HIER DIE RISIKOEINSTELLUNG HÖHER GEWORDEN SEIN - ABER EBEN EXTREMER GETRENNT?? HABEN WIR SEPARATE MITTELWERTE FÜR before AND after PRO LAND?]



To summarize the findings. The relationship of risk perception and the social factors reveals clearly that after the accident men, older people,  people with better education and those who trust more in people changed their risk perception [STÄRKER] more drastically. Therefore the perceived differences in risk perception decline after the accident. This results are in line with the adaptation-after-risk-hypothesis. There is one effect that shows the opposite. Individuals who trust in governance after the accident did not adapt their risk perception and now show an even stronger decline in risk perception. 







 Conclusion:  

This paper contributes to research on the effect of a nuclear accident on nuclear risk perception  in the population. Before the accident men, elder people and well educated people tend to have lower risk perception. We therefore claim that there is a risk perception gap within the society itself and not only   between expert and laypeople as mentioned in previous research (e.g. Slovic et al 1982). The heterogeneous distribution of risk perception reveals that  more discriminated or unprivileged people with less status express higher risk perception on nuclear power. This results are in line with previous research (e.g. Pampel 2011). 
One possible explanation is that people who support nuclear power are more convinced that natural risks can be controlled by technology (Flynn et al. 1994) and moreover that innovation and technology is necessary to protect economic development and wealth [XXX BELEG].   




Our main question is to analyze how the nuclear accident changed the pre-Fukushima perception gap. Do we observe a change towards a widening risk perception gap or does the perception gap shrink after the accident? Is there more radicalization or more Adaptation/solidarity [WIE NENNEN WIR DAS DING? WAS IST DAS GEGENTEIL VON ADAPTATION?]? Comparing risk perception before and after the accident we can prove the `adaptation'-hypothesis [SOLLEN WIR SIE SO NENNEN?] that after the accident peoples's risk perception changed towards a more homogenized effect within population. We assume that after the accident everybody perceived higher risks but the effect depend on the pre Fukshima level. More optimistic  people changed their risk perception more drastically and got more concerned after the accident  than people with a priori high risk perception, who's opinion was not changed by only affirmed by the accident. 



What are our implications for policymakers? Overall we observe a high level of risk-perception among individuals implying a high threat potential in case of a nuclear accident. Because of the more homogenized and higher level of risk perception in the population after an accident there is a high chance that support for governmental decisions will decline. An accident also destroys trust in governmental institutions. To gain back  trust in their capacity to act, the national government than needs a immediate strategy to implement new energy related policies. Thus policies like the [ATOMAUSSTIEG] in Germany or Switzerland was not a irrational and exaggerated decision but a logical consequence to quieten fears of future accidents in society and to prove government's capacity to act. More research on cross national comparison on risk-perception is necessary to explain, why in some countries like the UK or France there was no public pressure to change their energy policy [ENERGIEPOLITIK??].



Our results supporting previous research (e.g. Slovic et al, 1982 \citep{Slovic:1982vr}) that states that population's perceived risks are not irrational, even if laypeople have on average express higher risk perception than experts. Of course experts' risk evaluation always includes the possibility of an accident (even if the likelihood is very low - an accident is possible at any time now or in the future), no matter how small the likelihood is. People don't perceive likelihoods they base their evaluation on observed incidents and rather think in relative frequencies based on accessible information than on likelihoods derived from hypothetical models (Gigerenzer and Hoffrage, 1997   \citep{gigerenzer_how_1995}). The objective fact of two major accidents since 1986 clearly adds evidence to individuals risk judgement and leads to the question [WIRFT DIE FRAGE AUF] whether population's perceived risk should taken into consideration for future risk evaluation processes.  Because laypeople don't know how to deal with abstract probabilistic models and focus on the negative objective facts it could be the case that's them who have a more realistic perception of the probability of an accident than experts. After Fukushima the chance of an accident in a year is 2/30 [BIN MIR NICHT SICHER OB MEINE LOGIK STIMMT?].  
 

Our results suggest that laypeople should somehow be included in the decision-making process. Even if their perception might be quite different from experts. Expert only can make a risk judgement based on their current knowledge, combined with uncertainty assumptions about their knowledge. Experts also can decide what parameters they include in their models - so models are a reduction of reality, never the reality itself. The decision how to deal with the uncertainty [RESTRSIKO]  remains a political decision. It is an open question, whether citizens should be included in the decision making process of using nuclear energy. 

                               

Some limitations of the present study need to be addressed. Because of limited questions in the survey  we focused only on perceived risk of nuclear power. As other studies mention (Siegrist et al, 2014; Whitfield et al, 2009), beside risk perception also perceived benefits of nuclear power shapes individual's acceptance of nuclear technology. The model we [AUFSTELLEN] also has limitations because  change in individual's risk perception is modeled simply in change of the probability assumption of a future accident. Beside the updated probability assumption, changes in risk perception could also be influenced because of new information of the damage potential after an accident. There are also some methodological limitation.  The ISSP study is based on a [Querschnittssample] sample not on longitudinal data. The methods we use allow us only to measure causal effects on the aggregated level for population's subgroups and not individuals. We therefore focus in this study on the effect for gender, age, education, income and trust. The ISSP data have limited questions on values and moral norms and can not contribute to studies focusing more value oriented models (e.g. de Groot and Steg, 2010; Flynn et al. 1994  \citep{deGroot:2010ez}). The results we use are based only on a short period before and after the accident. We therefore can't test if the observed change in risk-perception is a long-term change. Renn (1990) shows that Chernobyl only had a short-term impact, thats also possible for the Fukushima accident. Our research reveals overall pattern of nuclear risk perception in the population. The results suggest plausible explanations why individuals react differently to the accident and why this changes could have influenced policy decisions after 3/11.  Further research could examine in more detail, why in some countries the Fukushima disaster led to political consequences and why in some not. More research on cross national differences in environmental attitudes, energy policies and environmental movements is necessary to draw a more detailed picture. 
             

No matter, how save nuclear technology is according to political risk assessment. In the nuclear world of complex nature-human-technology interaction there will always be a factor, that can't be controlled for. Therefore hazards like nuclear technology will always be perceived as dangerous. Risk perception on nuclear technology will be an important topic in the future also because we also have to deal with the consequences of the use of nuclear technology - most of all the nuclear waste disposal, with unknown consequences in the future. More investment in more sustainable technologies is necessary to be prepared for the political consequences that will change peoples opinion after the next nuclear disaster will happen. 
 
    

   







% Research question. 


% Conclusion
% In diesem Paper verstehen wir  Risk Perception nicht im traditionellen Sinne (als Funktion von objektiven Wsk und den Folgen), sondern Betrachten public concern of a risk more widly. Die Reaktionen in der Bevölkerung zeichnen ein Bild der Risikowahrnehmung, das die Senibilisierung zu Sozialen, Technischen udn Psychologischen Faktoren darstellt. (Slovic, 2000, 392) 

%  (!) Public concern is not ignorant or irrational. Public's reactions to risk can be attributed to a sensitivity to  technical, social, and psychological qualities of hazards that are not well-modeled in technical risk-assessment (e.g. qualities such as uncertainty in risk-assessment, perceived inequity (=Ungerechtigkeit) in the distribution of risks and benefits, and aversion to being exposed to risks that are involuntary not under one's control or dreaded (=gefürchtet)). (Slovic, 2000, p392)

% "We do not expect that the accident will have long-lasting effects on the attitudes formed by teenagers or young adults. The discussion about the benefits of renewables may have a negative impact on the benefit perception of nuclear power. Therefore, we cannot rule out that future generations who will start to participate in the political process in the next few years may have more negative attitudes toward nuclear power com- pared with the sample examined in the present study." (Siegrist 2014 p.6)  



%%%%%%%%%%%%% TEXTBAUSTEINE %%%%%%%%%%%%%%%%%%%%%%%
% The aim of the present study was to examine      
% to shape people's opinion                        
% Our results suggest that
% Some limitations of the present study need to be addressed.
% The present study focused on (the impact of perceived risk and perceived benefit on the acceptance of nuclear power.) 
% The study may thus provide more insights into the complex affect-laden imagery of nuclear power plants
% This can best be explained by... 
%  adjust views when confronted with new information
% Nuclear-power-related risks can be described as low-probability, high-costs events that impact not only the environment, but also society collectively with repercussions (Auswirkungen) beyond the individual. (Costa Font p1274)                                                      
% We argue that
% A further objective of this article is to examine whether 
% This study has examined the influence of ... on ... 
% Examination of these survey data allow us to test whether there is evidence
% We also find that 
% This finding signals the need
% As for control variables, we have found evidence of the influence of gender on affecting attitudes but not in the direction expected. However,... 
% A number of hypothesis have been put forward to explain...

%%%%%%%%%%%%%% To Do %%%%%%%%%%%%%%%%%%%%%%%%%%%%%%%

%  





